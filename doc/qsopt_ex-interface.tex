% generated by GAPDoc2LaTeX from XML source (Frank Luebeck)
\documentclass[a4paper,11pt]{report}

\usepackage{a4wide}
\sloppy
\pagestyle{myheadings}
\usepackage{amssymb}
\usepackage[utf8]{inputenc}
\usepackage{makeidx}
\makeindex
\usepackage{color}
\definecolor{FireBrick}{rgb}{0.5812,0.0074,0.0083}
\definecolor{RoyalBlue}{rgb}{0.0236,0.0894,0.6179}
\definecolor{RoyalGreen}{rgb}{0.0236,0.6179,0.0894}
\definecolor{RoyalRed}{rgb}{0.6179,0.0236,0.0894}
\definecolor{LightBlue}{rgb}{0.8544,0.9511,1.0000}
\definecolor{Black}{rgb}{0.0,0.0,0.0}

\definecolor{linkColor}{rgb}{0.0,0.0,0.554}
\definecolor{citeColor}{rgb}{0.0,0.0,0.554}
\definecolor{fileColor}{rgb}{0.0,0.0,0.554}
\definecolor{urlColor}{rgb}{0.0,0.0,0.554}
\definecolor{promptColor}{rgb}{0.0,0.0,0.589}
\definecolor{brkpromptColor}{rgb}{0.589,0.0,0.0}
\definecolor{gapinputColor}{rgb}{0.589,0.0,0.0}
\definecolor{gapoutputColor}{rgb}{0.0,0.0,0.0}

%%  for a long time these were red and blue by default,
%%  now black, but keep variables to overwrite
\definecolor{FuncColor}{rgb}{0.0,0.0,0.0}
%% strange name because of pdflatex bug:
\definecolor{Chapter }{rgb}{0.0,0.0,0.0}
\definecolor{DarkOlive}{rgb}{0.1047,0.2412,0.0064}


\usepackage{fancyvrb}

\usepackage{mathptmx,helvet}
\usepackage[T1]{fontenc}
\usepackage{textcomp}


\usepackage[
            pdftex=true,
            bookmarks=true,        
            a4paper=true,
            pdftitle={Written with GAPDoc},
            pdfcreator={LaTeX with hyperref package / GAPDoc},
            colorlinks=true,
            backref=page,
            breaklinks=true,
            linkcolor=linkColor,
            citecolor=citeColor,
            filecolor=fileColor,
            urlcolor=urlColor,
            pdfpagemode={UseNone}, 
           ]{hyperref}

\newcommand{\maintitlesize}{\fontsize{50}{55}\selectfont}

% write page numbers to a .pnr log file for online help
\newwrite\pagenrlog
\immediate\openout\pagenrlog =\jobname.pnr
\immediate\write\pagenrlog{PAGENRS := [}
\newcommand{\logpage}[1]{\protect\write\pagenrlog{#1, \thepage,}}
%% were never documented, give conflicts with some additional packages

\newcommand{\GAP}{\textsf{GAP}}

%% nicer description environments, allows long labels
\usepackage{enumitem}
\setdescription{style=nextline}

%% depth of toc
\setcounter{tocdepth}{1}





%% command for ColorPrompt style examples
\newcommand{\gapprompt}[1]{\color{promptColor}{\bfseries #1}}
\newcommand{\gapbrkprompt}[1]{\color{brkpromptColor}{\bfseries #1}}
\newcommand{\gapinput}[1]{\color{gapinputColor}{#1}}


\begin{document}

\logpage{[ 0, 0, 0 ]}
\begin{titlepage}
\mbox{}\vfill

\begin{center}{\maintitlesize \textbf{ qsopt{\textunderscore}ex-interface \mbox{}}}\\
\vfill

\hypersetup{pdftitle= qsopt{\textunderscore}ex-interface }
\markright{\scriptsize \mbox{}\hfill  qsopt{\textunderscore}ex-interface  \hfill\mbox{}}
{\Huge \textbf{ An Interface to QSopt exact LP solver \mbox{}}}\\
\vfill

{\Huge  1.0 \mbox{}}\\[1cm]
{ 03/28/2016 \mbox{}}\\[1cm]
\mbox{}\\[2cm]
{\Large \textbf{ Jayant Apte\\
    \mbox{}}}\\
\hypersetup{pdfauthor= Jayant Apte\\
    }
\end{center}\vfill

\mbox{}\\
{\mbox{}\\
\small \noindent \textbf{ Jayant Apte\\
    }  Email: \href{mailto://jayant91089@gmail.com} {\texttt{jayant91089@gmail.com}}\\
  Homepage: \href{https://sites.google.com/site/jayantapteshomepage/} {\texttt{https://sites.google.com/site/jayantapteshomepage/}}\\
  Address: \begin{minipage}[t]{8cm}\noindent
 Department of Electrical and Computer Engineering\\
 Drexel University\\
 Philadelphia, PA 19104\\
 \end{minipage}
}\\
\end{titlepage}

\newpage\setcounter{page}{2}
\newpage

\def\contentsname{Contents\logpage{[ 0, 0, 1 ]}}

\tableofcontents
\newpage

 \index{\textsf{qsopt{\textunderscore}ex-interface}}     
\chapter{\textcolor{Chapter }{Introduction}}\label{Chapter_Introduction}
\logpage{[ 1, 0, 0 ]}
\hyperdef{L}{X7DFB63A97E67C0A1}{}
{
  $\texttt{qsopt}\_\texttt{ex-interface}$ is a GAP package that provides an interface to $QSopt$ exact rational linear program solver \cite{qs} by Applegate,Cook,Dash and Espinoza. This is a minimalist package exposing
parts of qsopt to GAP. The particular version of QSopt-exact solver this
package currently follows is 2.5.10-patch 3 of a fork of the original software
maintained by Jon Lund Steffenson \cite{qsjon}, which removes certain dependencies and makes the software easier to build. $\texttt{qsopt}\_\texttt{ex-interface}$ provides a C wrapper qsinterface.c to the solver. It is currently available
for Unix/Linux systems running GAP $4.5+$. }

   
\chapter{\textcolor{Chapter }{Installation}}\label{Chapter_Installation}
\logpage{[ 2, 0, 0 ]}
\hyperdef{L}{X8360C04082558A12}{}
{
  Assuming you already have GAP 4.5+ installed, you can follow the steps below
to install the package: 
\begin{itemize}
\item  To get the newest version of $\texttt{qsopt}\_\texttt{ex-interface}$, download the .zip archive from \href{https://github.com/jayant91089/qsopt_ex-interface} {\texttt{https://github.com/jayant91089/qsopt{\textunderscore}ex-interface}} and unpack it using '$\texttt{unzip qsopt}\_\texttt{ex-interface-x.zip}$' in the terminal. Do this preferably inside the $pkg$ subdirectory of your GAP 4 installation. It creates a subdirectory called $\texttt{qsopt}\_\texttt{ex-interface}$. If you do not know the whereabouts of the $pkg$ subdirectory, invoke the following in GAP: 
\end{itemize}
 
\begin{Verbatim}[fontsize=\small,frame=single,label=Code]
  GAPInfo.("RootPaths");
\end{Verbatim}
 Look for pkg directory inside any of the paths returned. 
\begin{itemize}
\item  Once unpacked, go to $\texttt{qsopt}\_\texttt{ex-interface}$ directory and run the install script $\texttt{unix-install.sh}$ from the terminal as $\texttt{sh unix-install.sh}$. This locally installs qsopt exact and its dependencies (GMP \cite{Granlund12},libz and libbz2) in lib and include folders. Alternatively, if you have
qsopt-exact and GMP already installed on your system, you can edit the
Makefile inside $\texttt{qsopt}\_\texttt{ex-interface}$ directory so that gcc finds the .so libraries. In latter case, you must
manually '$\texttt{make all}$' from the terminal inside $\texttt{qsopt}\_\texttt{ex-interface}$ directory. 
\item  Above step creates an executable
\texttt{\symbol{92}}texttt\texttt{\symbol{123}}qsi\texttt{\symbol{125}} inside
the $\texttt{qsopt}\_\texttt{ex-interface}$ directory, which serves as the interface. Note that before using the package
in GAP, one must edit either the environment variable $\texttt{LD}\_\texttt{LIBRARY}\_\texttt{PATH}$ or the so that
\texttt{\symbol{92}}texttt\texttt{\symbol{123}}qsi\texttt{\symbol{125}} finds
the locally installed libraries. 
\item  One can now start using $\texttt{qsopt}\_\texttt{ex-interface}$ by invoking 
\end{itemize}
 
\begin{Verbatim}[fontsize=\small,frame=single,label=Code]
  LoadPackage( "qsopt_ex-interface");
\end{Verbatim}
 from within GAP. To expose more $\texttt{QSopt exact}$ functionality to GAP, one can extend the C part of the interface i.e. $\texttt{qsinterface.c}$. The relevent details of how the interface works are in $\texttt{qsinterface.c}$ itself. }

   
\chapter{\textcolor{Chapter }{Usage}}\label{Chapter_Usage}
\logpage{[ 3, 0, 0 ]}
\hyperdef{L}{X86A9B6F87E619FFF}{}
{
  
\section{\textcolor{Chapter }{Available functions}}\label{Chapter_Usage_Section_Available_functions}
\logpage{[ 3, 1, 0 ]}
\hyperdef{L}{X835D65A88292737E}{}
{
  In this section we shall look at the functions provided by
qsopt{\textunderscore}ex-interface. $\texttt{qsopt}\_\texttt{ex-interface}$ allows GAP to communicate with external LP solver process via a stream object
of category IsInputOutputStream(). This stream serves as a handle via which
one can load/solve/modify linear programs. Note that it is possible to
maintain several such steams (and hence LPs) at any given time. However, the
gap commands to solve/modify these LPs that are currently available in this
package are blocking functions. 

\subsection{\textcolor{Chapter }{LoadQSLP}}
\logpage{[ 3, 1, 1 ]}\nobreak
\hyperdef{L}{X7F3A2171848094DB}{}
{\noindent\textcolor{FuncColor}{$\triangleright$\ \ \texttt{LoadQSLP({\mdseries\slshape obj, A, b, linrows, qs{\textunderscore}exec, optargs})\index{LoadQSLP@\texttt{LoadQSLP}}
\label{LoadQSLP}
}\hfill{\scriptsize (function)}}\\
\textbf{\indent Returns:\ }
A list 



 This function loads an LP by invoking external qsopt-exact LP solver process.
It accepts following arguments: 
\begin{itemize}
\item  $obj$ - Objective function coefficients, provided as a list 
\item  $A$ - A list of lists corresponding to constraints 
\item  $b$ - Right hand side of constraints 
\item  $linrows$ - A list of indices of members of $A$ that are equalities 
\item  $qs\_exec$ - A string describing complete path to 'qsi' executable (including 'qsi') 
\end{itemize}
 Returns a list $[s,rval]$ where 's' is a gap object of category IsInputOutputStream() and 'rval' $=1/-1$ indicates success/failure. If 'rval=1', 's' is ready to be be used to solve
linear programs. }

 

\subsection{\textcolor{Chapter }{LoadQSLPobj}}
\logpage{[ 3, 1, 2 ]}\nobreak
\hyperdef{L}{X7914011C7AB692CF}{}
{\noindent\textcolor{FuncColor}{$\triangleright$\ \ \texttt{LoadQSLPobj({\mdseries\slshape s, obj})\index{LoadQSLPobj@\texttt{LoadQSLPobj}}
\label{LoadQSLPobj}
}\hfill{\scriptsize (function)}}\\
\textbf{\indent Returns:\ }
An integer 



 This function loads a new objective. It accepts following arguments: 
\begin{itemize}
\item  $s$ - gap object of category IsInputOutputStream(), handle to an already loaded LP 
\item  $obj$ - Objective function coefficients, provided as a list 
\end{itemize}
 Returns an integer 'rval' $=1/-1$ that indicate success/failure. If 'rval=1', the LP associated with 's' is
successfully modified. }

 

\subsection{\textcolor{Chapter }{SolveQSLP}}
\logpage{[ 3, 1, 3 ]}\nobreak
\hyperdef{L}{X7D6238047D53BB97}{}
{\noindent\textcolor{FuncColor}{$\triangleright$\ \ \texttt{SolveQSLP({\mdseries\slshape s, optargs})\index{SolveQSLP@\texttt{SolveQSLP}}
\label{SolveQSLP}
}\hfill{\scriptsize (function)}}\\
\textbf{\indent Returns:\ }
An integer 



 This function solves an LP by invoking external qsopt-exact LP solver process.
It accepts following arguments: 
\begin{itemize}
\item  $s$ - gap object of category IsInputOutputStream(), handle to an already loaded LP 
\item  $optargs$ - A list of optional arguments. Currently supports only one optional argument,
which is an integer specifying simplex variant to use: $optargs=[1]$ for primal simplex, $optargs=[2]$ for dual simplex and $optargs=[3]$ for either 
\end{itemize}
 Returns an integer $status$ that is the integer returned by $\texttt{mpq}\_\texttt{QSget}\_\texttt{status}()$ function. }

 

\subsection{\textcolor{Chapter }{FlushQSLP}}
\logpage{[ 3, 1, 4 ]}\nobreak
\hyperdef{L}{X7E4E00C884F72A22}{}
{\noindent\textcolor{FuncColor}{$\triangleright$\ \ \texttt{FlushQSLP({\mdseries\slshape s})\index{FlushQSLP@\texttt{FlushQSLP}}
\label{FlushQSLP}
}\hfill{\scriptsize (function)}}\\
\textbf{\indent Returns:\ }




 This function terminates the external processes associated with given LP
handle. It accepts following arguments: 
\begin{itemize}
\item  $s$ - gap object of category IsInputOutputStream(), handle to an already loaded LP 
\end{itemize}
 Returns Nothing }

 

\subsection{\textcolor{Chapter }{GetQSLPsol{\textunderscore}primal}}
\logpage{[ 3, 1, 5 ]}\nobreak
\hyperdef{L}{X79D8E9C482C75CC8}{}
{\noindent\textcolor{FuncColor}{$\triangleright$\ \ \texttt{GetQSLPsol{\textunderscore}primal({\mdseries\slshape s})\index{GetQSLPsolprimal@\texttt{GetQSLPsol{\textunderscore}primal}}
\label{GetQSLPsolprimal}
}\hfill{\scriptsize (function)}}\\
\textbf{\indent Returns:\ }
A list 



 This function obtains the primal solution along with the associated vertex
vertex, for the most recently solved LP. It accepts following arguments: 
\begin{itemize}
\item  $s$ - gap object of category IsInputOutputStream(), handle to an already loaded LP 
\end{itemize}
 Returns A list $[status,val\_rval,val,x\_rval,x]$ if optimal solution exists and a list $[status]$ otherwise. If $status=1$, $val\_rval$ and $x\_rval$ indicate validity of $val$ and $x$ (valid if $1$ and invalid if $-1$) which are optimal solution and (primal) vertex achieving optimal solution
respectively. Other status values correspond to the integer returned by $\texttt{mpq}\_\texttt{QSget}\_\texttt{status}()$ function. }

 

\subsection{\textcolor{Chapter }{GetQSLPsol{\textunderscore}dual}}
\logpage{[ 3, 1, 6 ]}\nobreak
\hyperdef{L}{X83D6C81084CFA840}{}
{\noindent\textcolor{FuncColor}{$\triangleright$\ \ \texttt{GetQSLPsol{\textunderscore}dual({\mdseries\slshape s})\index{GetQSLPsoldual@\texttt{GetQSLPsol{\textunderscore}dual}}
\label{GetQSLPsoldual}
}\hfill{\scriptsize (function)}}\\
\textbf{\indent Returns:\ }
A list 



 This function obtains the primal solution along with the associated vertex
vertex, for the most recently solved LP. It accepts following arguments: 
\begin{itemize}
\item  $s$ - gap object of category IsInputOutputStream(), handle to an already loaded LP 
\end{itemize}
 Returns A list $[status,val\_rval,val,y\_rval,y]$ if optimal solution exists and a list $[status]$ otherwise. If $status=1$, $val\_rval$ and $x\_rval$ indicate validity of $val$ and $x$ (valid if $1$ and invalid if $-1$) which are optimal solution and (dual) vertex achieving optimal solution
respectively. Other status values correspond to the integer returned by $\texttt{mpq}\_\texttt{QSget}\_\texttt{status}()$ function. }

 

\subsection{\textcolor{Chapter }{ChangeQSrhs}}
\logpage{[ 3, 1, 7 ]}\nobreak
\hyperdef{L}{X7A09A26783695F12}{}
{\noindent\textcolor{FuncColor}{$\triangleright$\ \ \texttt{ChangeQSrhs({\mdseries\slshape s, row, coef})\index{ChangeQSrhs@\texttt{ChangeQSrhs}}
\label{ChangeQSrhs}
}\hfill{\scriptsize (function)}}\\
\textbf{\indent Returns:\ }
An integer 



 This function changes the value of single rhs coefficient in specified row. It
accepts following arguments: 
\begin{itemize}
\item  $s$ - gap object of category IsInputOutputStream(), handle to an already loaded LP 
\item  $row$ - row index of the inequility whose rhs is to be changed 
\item  $coef$ - new rhs coefficient 
\end{itemize}
 Returns A an integer which is itself returned by QSopt$\_$ex function $\texttt{mpq}\_\texttt{QSchange}\_\texttt{rhscoef}$ }

 

\subsection{\textcolor{Chapter }{DelQSrow}}
\logpage{[ 3, 1, 8 ]}\nobreak
\hyperdef{L}{X78E2B3C1859938AB}{}
{\noindent\textcolor{FuncColor}{$\triangleright$\ \ \texttt{DelQSrow({\mdseries\slshape s, row})\index{DelQSrow@\texttt{DelQSrow}}
\label{DelQSrow}
}\hfill{\scriptsize (function)}}\\
\textbf{\indent Returns:\ }
An integer 



 This function deletes the specified row. (Note that for repeated use, one must
relabel rows as QSopt$\_$ex would treat eg. the second row as first row if we delete the first row) It
accepts following arguments: 
\begin{itemize}
\item  $s$ - gap object of category IsInputOutputStream(), handle to an already loaded LP 
\item  $row$ - row index of the inequility whose rhs is to be changed 
\end{itemize}
 Returns A an integer which is itself returned by QSopt$\_$ex function $\texttt{mpq}\_\texttt{QSchange}\_\texttt{rhscoef}$ }

 

\subsection{\textcolor{Chapter }{ChangeQSsense}}
\logpage{[ 3, 1, 9 ]}\nobreak
\hyperdef{L}{X7E5E3CAB7C48B062}{}
{\noindent\textcolor{FuncColor}{$\triangleright$\ \ \texttt{ChangeQSsense({\mdseries\slshape s, row, coef})\index{ChangeQSsense@\texttt{ChangeQSsense}}
\label{ChangeQSsense}
}\hfill{\scriptsize (function)}}\\
\textbf{\indent Returns:\ }
An integer 



 This function changes the sense (equality or inequality) of a particular row.
It accepts following arguments: 
\begin{itemize}
\item  $s$ - gap object of category IsInputOutputStream(), handle to an already loaded LP 
\item  $row$ - row index of the inequility whose sense is to be changed 
\item  $newsense$ - A single character string describing the new sense, "L" for $\leq$ and "E" for $=$ 
\end{itemize}
 Returns An integer which is itself returned by QSopt$\_$ex function $\texttt{mpq}\_\texttt{QSchange}\_\texttt{sense}$ }

 

\subsection{\textcolor{Chapter }{ChangeQScoef}}
\logpage{[ 3, 1, 10 ]}\nobreak
\hyperdef{L}{X7E5D355C8021531E}{}
{\noindent\textcolor{FuncColor}{$\triangleright$\ \ \texttt{ChangeQScoef({\mdseries\slshape s, row, coef})\index{ChangeQScoef@\texttt{ChangeQScoef}}
\label{ChangeQScoef}
}\hfill{\scriptsize (function)}}\\
\textbf{\indent Returns:\ }
An integer 



 This function changes a particular coefficient in the constraint matrix. It
accepts following arguments: 
\begin{itemize}
\item  $s$ - gap object of category IsInputOutputStream(), handle to an already loaded LP 
\item  $row$ - row index of the inequility to which the coefficient to be changed belongs 
\item  $col$ - column index of the inequility whose sense is to be changed 
\item  $coef$ - A rational number or an integer 
\end{itemize}
 Returns A an integer which is itself returned by QSopt$\_$ex function $\texttt{mpq}\_\texttt{QSchange}\_\texttt{sense}$ }

 

\subsection{\textcolor{Chapter }{DisplayLPQS}}
\logpage{[ 3, 1, 11 ]}\nobreak
\hyperdef{L}{X7B47C76B791F69E3}{}
{\noindent\textcolor{FuncColor}{$\triangleright$\ \ \texttt{DisplayLPQS({\mdseries\slshape s})\index{DisplayLPQS@\texttt{DisplayLPQS}}
\label{DisplayLPQS}
}\hfill{\scriptsize (function)}}\\
\textbf{\indent Returns:\ }
Nothing 



 This function displays an already loaded LP. It accepts following arguments: 
\begin{itemize}
\item  $s$ - gap object of category IsInputOutputStream(), handle to an already loaded LP 
\end{itemize}
 Returns Nothing }

 }

 
\section{\textcolor{Chapter }{Example}}\label{Chapter_Usage_Section_Example}
\logpage{[ 3, 2, 0 ]}
\hyperdef{L}{X85861B017AEEC50B}{}
{
  Following example explains the standard workflow with qsopt $\texttt{qsopt}\_\texttt{ex-interface}$. We show how to load, solve, display and modify a linear program. 
\begin{Verbatim}[commandchars=!@|,fontsize=\small,frame=single,label=Example]
  !gapprompt@gap>| !gapinput@#  absolute path to the interface executable|
  !gapprompt@>| !gapinput@qs_exec:="/home/aspitrg3-users/jayant/qsopt_interface/dummy";;|
  !gapprompt@gap>| !gapinput@# Construt a 3-D cube|
  !gapprompt@>| !gapinput@A:=[[1,0,0],[0,1,0],[0,0,1],[-1,0,0],[0,-1,0],[0,0,-1]];;|
  !gapprompt@gap>| !gapinput@b:=[1,1,1,0,0,0];;|
  !gapprompt@gap>| !gapinput@rlist:=LoadQSLP([1,1,1],A,b,[],qs_exec);;|
  !gapprompt@gap>| !gapinput@rlist[1]; # stdin/stdout handle to the loaded LP|
  < input/output stream to dummy >
  !gapprompt@gap>| !gapinput@s:=rlist[1];;|
  !gapprompt@gap>| !gapinput@DisplayLPQS(s);|
  Problem
   prob
  Maximize
   obj:   c0 +  c1 +  c2
  Subject To
   r0:   c0 <= 1
   r1:   c1 <= 1
   r2:   c2 <= 1
   r3:  -  c0 <= 0
   r4:  -  c1 <= 0
   r5:  -  c2 <= 0
  Bounds
   c0 free
   c1 free
   c2 free
  End
  !gapprompt@gap>| !gapinput@SolveQSLP(s,[]); # returns status, 1 for success|
  1
  !gapprompt@gap>| !gapinput@rlist:=GetQSLPsol_primal(s);; # get primal solution|
  !gapprompt@gap>| !gapinput@rlist[1]; # return status|
  1
  !gapprompt@gap>| !gapinput@rlist[2]; # val_rval, 0 means sane|
  0
  !gapprompt@gap>| !gapinput@rlist[3]; # val, LP solution|
  3
  !gapprompt@gap>| !gapinput@rlist[4]; # x_rval, 0 means sane|
  0
  !gapprompt@gap>| !gapinput@rlist[5]; # x, optimum vertex|
  [ 1, 1, 1 ]
  !gapprompt@gap>| !gapinput@rlist:=GetQSLPsol_dual(s);;  #  get dual solution|
  !gapprompt@gap>| !gapinput@rlist[1]; # status|
  1
  !gapprompt@gap>| !gapinput@rlist[2]; # val_rval|
  0
  !gapprompt@gap>| !gapinput@rlist[3]; # val|
  3
  !gapprompt@gap>| !gapinput@rlist[4]; # y_rval|
  0
  !gapprompt@gap>| !gapinput@rlist[5]; # y|
  [ 1, 1, 1, 0, 0, 0 ]
  !gapprompt@gap>| !gapinput@LoadQSLPobj(s,[-1,-1,-1]); # to minimize, negate the objective|
  1
  !gapprompt@gap>| !gapinput@SolveQSLP(s,[]); # returns status, 1 for success|
  1
  !gapprompt@gap>| !gapinput@rlist:=GetQSLPsol_primal(s);  #  get primal solution|
  [ 1, 0, 0, 0, [ 0, 0, 0 ] ]
  !gapprompt@gap>| !gapinput@ChangeQSsense(s,1,"E"); # tighten first inequality (r0)|
  0
  !gapprompt@gap>| !gapinput@DisplayLPQS(s);|
  Problem
  prob
  Maximize
   obj:  -  c0 -  c1 -  c2
  Subject To
   r0:   c0 = 1
   r1:   c1 <= 1
   r2:   c2 <= 1
   r3:  -  c0 <= 0
   r4:  -  c1 <= 0
   r5:  -  c2 <= 0
  Bounds
   c0 free
   c1 free
   c2 free
  End
  !gapprompt@gap>| !gapinput@ChangeQSrhs(s,1,3/2); # change first row r0's rhs to 3/2|
  0
  !gapprompt@gap>| !gapinput@DisplayLPQS(s);|
  Problem
  prob
  Maximize
  obj:  -  c0 -  c1 -  c2
  Subject To
   r0:   c0 = 3/2
   r1:   c1 <= 1
   r2:   c2 <= 1
   r3:  -  c0 <= 0
   r4:  -  c1 <= 0
   r5:  -  c2 <= 0
  Bounds
   c0 free
   c1 free
   c2 free
  End
  !gapprompt@gap>| !gapinput@SolveQSLP(s,[]); # returns status, 1 for success|
  1
  !gapprompt@gap>| !gapinput@rlist:=GetQSLPsol_primal(s);  #  get primal solution|
  [ 1, 0, -3/2, 0, [ 3/2, 0, 0 ] ]
  !gapprompt@gap>| !gapinput@DelQSrow(s,1); # delete the first row|
  0
  !gapprompt@gap>| !gapinput@DisplayLPQS(s);|
  Problem
  prob
  Maximize
   obj:  -  c0 -  c1 -  c2
  Subject To
   r1:   c1 <= 1
   r2:   c2 <= 1
   r3:  -  c0 <= 0
   r4:  -  c1 <= 0
   r5:  -  c2 <= 0
  Bounds
   c0 free
   c1 free
   c2 free
  End
\end{Verbatim}
 }

 }

 \def\bibname{References\logpage{[ "Bib", 0, 0 ]}
\hyperdef{L}{X7A6F98FD85F02BFE}{}
}

\bibliographystyle{alpha}
\bibliography{qsopt_ex-interface.bib}

\addcontentsline{toc}{chapter}{References}

\def\indexname{Index\logpage{[ "Ind", 0, 0 ]}
\hyperdef{L}{X83A0356F839C696F}{}
}

\cleardoublepage
\phantomsection
\addcontentsline{toc}{chapter}{Index}


\printindex

\newpage
\immediate\write\pagenrlog{["End"], \arabic{page}];}
\immediate\closeout\pagenrlog
\end{document}
